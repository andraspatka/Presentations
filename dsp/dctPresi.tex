\documentclass{beamer}
\usetheme{metropolis}

\usepackage{graphicx}
\usepackage{appendixnumberbeamer}

\title{DCT - Diszkrét koszinusz-transzformáció}
\author{Patka Zsolt-András}
\institute{Sapientia EMTE - Számítástechnika IV}
\date{2020.03.09}

\begin{document}

\begin{frame}
    \titlepage
\end{frame}

\begin{frame}
\frametitle{Miről lesz szó?}
\tableofcontents
\end{frame}

\section{Adattömörítés}

\subsection{Veszteségmentes adattömörítés}
\begin{frame}{\secname : \subsecname}
\begin{enumerate}
    \item Lehetséges a tömörített adatból az eredeti adatok pontos rekonstrukciója
    \item Huffman
    \beging{itemize}
    \item változó hosszuságú kódok
    \item gyakrabban előfordult szimbólumok rövidebb kódot kapnak
    \end{itemize}
    \item Például: Szöveges állományok tömörítése
\end{enumerate}    
\end{frame}

\subsection{Veszteséges adattömörítés}
\begin{frame}{\secname : \subsecname}
\begin{enumerate}
    \item A tömörített adatból az eredeti adatok nem pontos rekonstrukcióját teszi lehetővé
    \item Például: képek (JPEG), videók tömörítése (MPEG, H.26X)
\end{enumerate}    
\end{frame}

\section{DCT}

\subsection{Fourier-transzformáció}
\begin{frame}{\secname : \subsecname}
    \begin{itemize}
        \item Periódikus jel felírása szinuszok és koszinuszok kombinációjaként
        \item $ X_k = \sum_{n=0}^{N - 1} x_n * [\cos(\frac{2 \pi}{N} kn) - i * sin(\frac{2\pi}{N} kn)]$
    \end{itemize}
\end{frame}

\subsection{DCT}
\begin{frame}{\secname : \subsecname}
    \begin{itemize}
        \item DCT-I:
        \begin{itemize}
            \item DCT-I: $ X_k = \frac{1}{2} (x_0 + (-1)^kx_{N-1}) + \sum_{n=0}^{N - 1} x_n * \cos[\frac{\pi}{N - 1} nk]$
        \end{itemize}
        \item DCT-II:
        \begin{itemize}
            \item DCT-II: $ X_k = \sum_{n=0}^{N - 1} x_n * \cos[\frac{\pi}{N} * (n + \frac{1}{2})k] $ $k=0,...,N-1$
            \item leggyakrabban használt forma
        \end{itemize}
    \end{itemize}
    
\end{frame}


\section{DCT alkalmazása - JPEG}

\subsection{Brainstorming}
\begin{frame}{\secname : \subsecname}

A brainstorming (ötletbörze) lényege az, hogy a gondolkodási folyamat ne legyen megszakítva.

Moderátor: \textbf{ajánlott}

Szabályai:
\begin{enumerate}
    \item fókuszálj a mennyiségre
    \item ne kritizálj
    \item osszd meg az ötleteidet, még akkor is ha ezek szokatlanok
    \item kombináld és javítsd az ötleteket
\end{enumerate}


\begin{frame}
\centering
{\Huge Köszönöm a figyelmet!}
\end{frame}

\appendix{Források}
\begin{footnotesize}
    \begin{itemize}
        \item https://en.wikipedia.org/wiki/JPEG#Discrete_cosine_transform
        \item https://www.youtube.com/watch?v=_bltj_7Ne2c
        \item https://www.youtube.com/watch?v=n_uNPbdenRs
        \item https://www.youtube.com/watch?v=Q2aEzeMDHMA&list=RDCMUC9-y-6csu5WGm29I7JiwpnA&start_radio=1&t=6
        \item https://hu.wikipedia.org/wiki/Vesztes\%C3\%A9gmentes_t\%C3\%B6m\%C3\%B6r\%C3\%ADt\%C3\%A9s
    \end{itemize}
\end{footnotesize}

\end{document}