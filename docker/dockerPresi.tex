\documentclass{beamer}
\usetheme{metropolis}

\usepackage{graphicx}
\usepackage{appendixnumberbeamer}

\title{Docker, CI, Travis}
\author{Patka Zsolt-András}
\institute{Sapientia - Computer Science BSc.}
\date{2020.03.16}

\begin{document}

\begin{frame}
    \titlepage
\end{frame}

\begin{frame}
\frametitle{What is this presentation about?}
\tableofcontents
\end{frame}

\section{Continous Integration}

\subsection{CI \/ CD}
\begin{frame}{\secname : \subsecname}
Continous Integration \/ Continous Delivery
    \begin{enumerate}
        \item Plan
        \item Code
        \item Build
        \item Test
        \item Release
        \item Deploy
        \item Operate
        \item Measure
        \item Repeat
    \end{enumerate}
\end{frame}


\section{Docker}

\subsection{Dockerfile}
\begin{frame}{\secname : \subsecname}

\begin{itemize}
    \item How should our image look like?
    \item Analog: Source code for Windows
\end{itemize}
\end{frame}

% TODO: Insert picture

\subsection{Docker image}
\begin{frame}{\secname : \subsecname}

% TODO: Insert picture

\begin{itemize}
    \item Result of building the Dockerfile
    \item Contains everything needed to spin up your application
    \item Analog: CD containing Windows
\end{itemize}

\end{frame}

\subsection{Docker container}
\begin{frame}{\secname : \subsecname}

% TODO: Insert picture

\begin{itemize}
    \item Result of starting a docker image
    \item Is alive, runs your application
    \item Analog: Your PC after you've installed Windows
\end{itemize}

\end{frame}


\section{How to: CI}

\subsection{Travis}
\begin{frame}[fragile]{\secname : \subsecname}

% TODO: Insert example .travis.yml file

\begin{columns}
\begin{column}{0.3\textwidth}
\begin{enumerate}
    \item Free (has a paid version as well)
    \item Can easily be integrated with Github
    \item Uses Docker containers internally
    \item .travis.yml: contains build steps to be executed on each push
\end{enumerate}
\end{column}

\begin{column}{0.7\textwidth}

\begin{scriptsize}
\begin{verbatim}

language: java
services:
 - docker
install: true
os: linux
dist: trusty
jdk: openjdk8
before_script: cd backend
script:
 - ./gradlew build && 
 ./gradlew test && ./gradlew bootJar
after_success:
 - docker build -t demobackend .
 - docker tag demobackend 
 "$DOCKER_USERNAME"/demobackend:latest
 - echo "$DOCKER_PASSWORD" | 
 docker login -u "$DOCKER_USERNAME" --password-stdin
 - docker push "$DOCKER_USERNAME"/demobackend

\end{verbatim}
\end{scriptsize}

\end{column}
\end{columns}
\end{frame}

\subsection{Github Actions}
\begin{frame}[fragile]{\secname : \subsecname}

% TODO: Insert example .github yml file
\begin{columns}
\begin{column}{0.3\textwidth}

\begin{enumerate}
    \item Free
    \item It's trivial to integrate it with Github
    \item Uses Docker containers internally
    \item Easy to use
\end{enumerate}
\end{column}

\begin{column}{0.7\textwidth}

\begin{tiny}
\begin{verbatim}
on:
    push:
    branches: 
        - develop
        - master
    pull_request:
    branches: 
        - master
jobs:
    build:
    runs-on: windows-latest
    steps:
    - uses: actions/checkout@v1
    - name: Set up Python 3.6.8
        uses: actions/setup-python@v1
        with:
        python-version: 3.6.8
    - name: Install dependencies
        run: |
        python -m pip install --upgrade pip
    - name: Test with pytest
        working-directory: src
        run: |
        pip install pytest
        pytest

\end{verbatim}
\end{tiny}

\end{column}
\end{columns}
\end{frame}


\section{Live demo}

\subsection{Backend}
\begin{frame}{\secname : \subsecname}

\begin{itemize}
    \item Spring boot app
    \item Has two simple endpoints
    \item Travis automatically (on each push):
    \begin{enumerate}
        \item Runs the build
        \item Runs the tests
        \item Creates a .jar file
        \item Build the Dockerfile
        \item Pushes the Docker image to DockerHub
    \end{enumerate}
\end{itemize}

\end{frame}

\subsection{Frontend}
\begin{frame}{\secname : \subsecname}

\begin{itemize}
    \item React
    \item Can send a request to the backend
    \item Travis automatically (on each push):
    \begin{enumerate}
        \item Runs the build
        \item Runs the tests
        \item Build the Dockerfile (two stage build)
        \item Pushes the Docker image to DockerHub
    \end{enumerate}
\end{itemize}

\end{frame}

\section{Advices}

\subsection{Docker commands cheatsheet}
\begin{frame}{\secname : \subsecname}
    You can find a cheat sheet containing commonly used Docker commands:

    \url{https://github.com/andraspatka/Presentations/blob/master/docker/cheatSheet.md}
\end{frame}


\begin{frame}
\centering
{\Huge Thank you for your attention!}
\end{frame}

\appendix{Sources}
\begin{footnotesize}
    \begin{itemize}
        \item https://mherman.org/blog/dockerizing-a-react-app/
    \end{itemize}
\end{footnotesize}

\end{document}