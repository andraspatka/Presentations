\documentclass{beamer}
\usetheme{Boadilla}

\title{Csoportos döntéshozatal}
\author{Patka Zsolt-András}
\institute{Sapientia EMTE}
\date{2020.03.09}

\begin{document}

\begin{frame}
    \titlepage
\end{frame}

\begin{frame}
\frametitle{Miről lesz szó?}
\tableofcontents
\end{frame}


\section{Bevezető}

\subsection{Sajátosságok}
\begin{frame}{\secname : \subsecname}
    \begin{itemize}
        \item nem egy ember hozza meg a döntést, hanem egy csoport
        \item minél bonyolultabb egy feladat, annál nagyobb hangsúlyt kap a csoportos döntés
        \item munkahelyi vezetés alsóbb szintjein - egyéni döntések
        \item munkahelyi vezetés felsőbb szintjein - csoportos döntések
    \end{itemize}
\end{frame}

\subsection{Fontossága}
\begin{frame}{\secname : \subsecname}
    \begin{itemize}
        \item azok akik a döntés által befolyásolva lesznek, gyakran kikényszerítik a részvételt
        \item könnyebb elfogadtatni a döntést (sajátjuknak tekintik a döntést)
        \item a csoport teljesítménye gyakran jobb, mint az átlagos csoporttagé
    \end{itemize}
\end{frame}


\subsection{Előnyök}
\begin{frame}{\secname : \subsecname}
    \begin{itemize}
        \item több ismeret, információ
        \item többféle probléma-megközelítés kerül felszínre.
        \item a csoport teljesítménye gyakran jobb, mint az átlagos csoporttagé
        \item a döntés elfogadtatása sokkal könnyebb
    \end{itemize}
\end{frame}

\subsection{Hátrányok}
\begin{frame}{\secname : \subsecname}
    \begin{itemize}
        \item hosszabb időt vesz igénybe
        \item előfordulhat, hogy egyetlen személy fogja uralni a folyamatot
        \item csoportnyomás
    \end{itemize}
\end{frame}


\section{Módszerek}

\subsection{Brainstorming}
\begin{frame}{\secname : \subsecname}

A brainstorming (ötletbörze) lényege az, hogy a gondolkodási folyamat ne legyen megszakítva. Ahogy jönnek az új ötletek, úgy lehet rájuk építeni,
ezeket feljavítani

Szabályai:
\begin{enumerate}
    \item fókuszálj a mennyiségre
    \item ne kritizálj
    \item furcsa ötletek 
    \item kombináld és javítsd az ötleteket
\end{enumerate}
\end{frame}

\subsection{6-3-5-ös módszer}
\begin{frame}{\secname : \subsecname}
    Stuff about brainstorming
\end{frame}

\subsubsection{Példa : 1/6}
\begin{frame}
    \begin{table}[H]
	\begin{center}
		\caption{Csokoládépudding - 635 módszer}
		\begin{tabular}{l|c|c|c}
		\textbf{Anna}   & új csomagolás 	                                                                & hűségpontok                                                           & kevesebb cukor \\
		\hline         
        \textbf{Gábor}  &  	& & \\
        \hline         
        \textbf{Attila} &   & & \\
        \hline         
        \textbf{László} &   & & \\
        \hline         
        \textbf{Kinga}  &   & & \\
        \hline         
        \textbf{Noémi}  &   & & \\
		\end{tabular}
	\end{center}
\end{table}
\end{frame}

\subsubsection{Példa : 6/6}
\begin{frame}
    \begin{table}[H]
	\begin{center}
        \caption{Csokoládépudding - 6-3-5-ös módszer}
        \scalebox{0.7}{
		\begin{tabular}{p{1.5cm}|p{5cm}|p{4cm}|p{4cm}}
		\textbf{Anna}   & új csomagolás 	                                                                & hűségpontok                                                           & kevesebb cukor \\
		\hline         
        \textbf{Gábor}  & bombon formájában 	                                                            & hűségkártya                                                           & hangsúlyozni az egészséges étkezést \\
        \hline         
        \textbf{Attila} & pudding bombonban       	                                                        & hűségpontok és hűségkártya az egész pudding termékvonalra             & az egészség jólléthez vezet, a csokoládé boldoggá tesz \\
        \hline         
        \textbf{László} & a cukorkákat nem kell hűvösen tartani $\rightarrow$ el lehet adni a terméket a kasszánál is  & díjak ha egy bizonyos pontmennyiség el van érve                         & figyelemfelhívó kampányok arról, hogy a csokoládé milyen egészséges \\
        \hline         
        \textbf{Kinga}  & gyűjtőalbumot lehetne adni a kasszánál a gyerekeknek       	                    & a pontokat be lehet váltani matricákra                                & kiegyensúlyozott, boldog és egészséges család a csokoládépudding fogyasztása által \\
        \hline         
        \textbf{Noémi}  & kitöltött gyűjtőalbumot be lehet cserélni egy ingyenes belépőre az állatkertbe    & az első 5 aki 1000 pontot összegyűjt, meg lesz hívva a puddinggyárba  & családi kirándulásokat finanszírozni \\
        \end{tabular}
        }
	\end{center}
\end{table}
\end{frame}

\subsection{Philips 66-os módszer}
\begin{frame}{\secname : \subsecname}
    Stuff about brainstorming
\end{frame}

\subsection{Edward de Bono 6 gondolkodó kalap}
\begin{frame}{\secname : \subsecname}
    Stuff about brainstorming
\end{frame}

\section{Alkalmazás}


\section*{Források}
\begin{frame}{\secname}
    \begin{itemize}
        \item Döntéselmélet, Szikora Péter, Óbudai Egyetem - Döntéselmélet – 2016
        \item \url{http://centroszet.hu/tananyag/vezetes/442_csoportos_dnts.html}
        \item \url{https://www.smithsonianmag.com/science-nature/spanx-on-steroids-how-speedo-created-the-new-record-breaking-swimsuit-9662/?c=y&story=fullstor}
        \item \url{https://penzugysziget.hu/index.php?option=com_content&view=article&id=2584:28tetel&catid=295&Itemid=401}
        \item InnovationNet: The art of creating and benefiting from innovation networks, Jan Kratzer
        \item https://axel-schroeder.de/kreativitaet-und-kreativitaetstechniken-fuer-handwerker-und-selbstaendige-methode-6-3-5/
    \end{itemize}
\end{frame}




\end{document}