\begin{table}[H]
	\begin{center}
        \caption{Csokoládépudding - 6-3-5-ös módszer}
        \scalebox{0.7}{
		\begin{tabular}{p{1.5cm}|p{4cm}|p{4cm}|p{4cm}}
		\textbf{Anna}   & új csomagolás 	                                                                & hűségpontok                                                           & kevesebb cukor \\
		\hline         
        \textbf{Gábor}  & bombon formájában 	                                                            & hűségkártya                                                           & hangsúlyozni az egészséges étkezést \\
        \hline         
        \textbf{Attila} & pudding bombonban       	                                                        & hűségpontok és hűségkártya az egész pudding termékvonalra             & az egészség jólléthez vezet, a csokoládé boldoggá tesz \\
        \hline         
        \textbf{László} & a cukorkákat nem kell hűvösen tartani $\rightarrow$ el lehet adni a terméket a kasszánál is  & díjak ha egy bizonyos pontmennyiség el van érve                         & figyelemfelhívó kampányok arról, hogy a csokoládé milyen egészséges \\
        \hline         
        \textbf{Kinga}  & gyűjtőalbumot lehetne adni a kasszánál a gyerekeknek       	                    & a pontokat be lehet váltani matricákra                                & kiegyensúlyozott, boldog és egészséges család a csokoládépudding fogyasztása által \\
        \hline         
        \textbf{Noémi}  & kitöltött gyűjtőalbumot be lehet cserélni egy ingyenes belépőre az állatkertbe    & az első 5 aki 1000 pontot összegyűjt, meg lesz hívva a puddinggyárba  & családi kirándulásokat finanszírozni \\
        \end{tabular}
        }
	\end{center}
\end{table}